\documentclass[11pt, leqno]{scrartcl}
\usepackage{polski}
\usepackage[polish]{babel}

\usepackage{graphicx, float, caption, subcaption}
\usepackage{tabularx, multirow, hyperref, enumitem}
\usepackage{listings, xcolor}
\usepackage{amsmath, amssymb}
\usepackage{algorithm}
\usepackage{algpseudocode}
%\usepackage{minted}

\definecolor{md-black}{rgb}{0.12, 0.12, 0.12}
\definecolor{md-teal}{rgb}{0.38, 0.79, 0.69}
\definecolor{md-mauve}{rgb}{0.76, 0.52, 0.75}
\definecolor{md-yellow}{rgb}{0.86, 0.86, 0.67}
\definecolor{md-green}{rgb}{0.13, 0.55, 0.13}
\definecolor{md-red}{rgb}{0.82, 0.10, 0.14}
\definecolor{md-gray}{rgb}{0.44, 0.46, 0.51}

\hypersetup{
    colorlinks=true,
    linkcolor=black,
    urlcolor=black,
    citecolor=black
}

\lstset{
    language=Python,
    basicstyle=\color{md-teal}\ttfamily,
    keywordstyle=\color{md-mauve},
    commentstyle=\color{md-green},
    stringstyle=\color{md-red},
    numbers=left,
    numberstyle=\small\color{md-gray}\ttfamily,
    stepnumber=1,
    numbersep=5pt,
    backgroundcolor=\color{md-black},
    showspaces=false,
    showstringspaces=false,
    showtabs=false,
    frame=none,
    tabsize=4,
    captionpos=b,
    breaklines=true,
    breakatwhitespace=false,
    escapeinside={\%*}{*)},
    numbersep=-10pt,
    morekeywords={as},
    classoffset=1,
    morekeywords={curve_fit},
    keywordstyle=\color{md-yellow},
    classoffset=0
}

\graphicspath{{../images/}}

\title{III zestaw zadań - Hierarchiczna kompresja macierzy}
\author{Kacper Kozubowski, Mateusz Podmokły \\ III
    rok Informatyka WI}
\date{27 listopad 2024}

\begin{document}
    \maketitle
    \section{Treść zadania}
    Proszę wybrać kolorową bitmapę, np 500 x 500, zamienić ją na
    3 macierze RGB (wartości z przedziału $[0,255]$)
    i zaimplementować:
    \begin{enumerate}
        \item Rekurencyjną kompresję macierzy z wykorzystaniem
            częściowego SVD
        \item Rysowanie skompresowanej macierzy
        \item Rysowanie skompresowanej bitmapy
    \end{enumerate}

    \section{Specyfikacja użytego środowiska}
    Specyfikacja:
    \begin{itemize}
        \item Środowisko: Jupyter Notebook,
        \item Język programowania: Python,
        \item System operacyjny: Microsoft Windows 11,
        \item Architektura systemu: x64.
    \end{itemize}

    \section{Działanie algorytmów}
    \subsection{Wykorzystane biblioteki}
    W realizacji rozwiązania wykorzystane zostały następujące
    biblioteki:
    \begin{lstlisting}
    import numpy as np
    import cv2
    import matplotlib.pyplot as plt
    import matplotlib.patches as patches
    \end{lstlisting}

    \subsection{Kod funkcji}
    \begin{figure}[H]
        \centering
        \includegraphics[width=1\linewidth]{Node.png}
        \caption{Struktura drzewa macierzy.}
    \end{figure}
    \begin{figure}[H]
        \centering
        \includegraphics[width=1\linewidth]{truncated_SVD.png}
        \caption{Częściowe SVD.}
    \end{figure}
    \begin{figure}[H]
        \centering
        \includegraphics[width=1\linewidth]{compress_matrix.png}
        \caption{Kompresja macierzy.}
    \end{figure}
    \begin{figure}[H]
        \centering
        \includegraphics[width=1\linewidth]{create_tree.png}
        \caption{Tworzenie drzewa kompresji.}
    \end{figure}

    \section{Wyniki algorytmów}
    \subsection{Singular Value Decomposition (SVD)}
    Wartości osobliwe dla każdego kanału RGB przykładowej bitmapy
    uzyskane z SVD.
    \begin{figure}[H]
        \centering
        \includegraphics[width=0.6\linewidth]{SVD_values.png}
        \caption{Wartości osobliwe kanałów RGB.}
    \end{figure}

    \subsection{Przykładowa bitmapa testowa}
    Nieskompresowana bitmapa w całości
    \begin{figure}[H]
        \centering
        \includegraphics[width=0.5\linewidth]{../zad3/bird.jpg}
        \caption{Obraz testowy.}
    \end{figure}
    \begin{figure}[H]
        \centering
        \includegraphics[width=1\linewidth]{bird_channels.png}
        \caption{Podział na kanały RGB.}
    \end{figure}

\end{document}
